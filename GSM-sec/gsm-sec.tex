\documentclass{beamer}

\usepackage[utf8]{inputenc}
\usepackage[T1]{fontenc}

\usepackage{color}
\usepackage{pgfpages}
\PassOptionsToPackage{hyphens}{url}
\usepackage{hyperref}
\usepackage{url}

\setbeameroption{show notes on second screen=right}

%document specific info
\title{Cell Phone Encryption}
\subtitle{``Anyone remember the Clipper Chip?''}
\author{Stephen ``ToxicSauce'' Walker-Weinshenker}
\institute{
  \inst{}
  Department of Computer Science\\
  Colorado State University
  \and
  \inst{}
  Department of Electrical and Computer Engineering\\
  Colorado State University
}
\date{\today}
%\subject{subject}

%beamer template
\definecolor{HDred}{RGB}{171,31,36}
\setbeamercolor{background canvas}{bg=gray}
\setbeamercolor{title}{fg=HDred}
\setbeamercolor{frametitle}{fg=HDred}
\setbeamercolor{logo}{bg=gray}

\logo{\includegraphics[height=1.5cm]{logo.png}}


% \setbeamertemplate{headline}
%   {
%
%   }

\useoutertheme[hideallsubsections]{sidebar}

\begin{document}
\frame{\titlepage}

% \begin{frame}
% \frametitle{Table of Contents}
% \tableofcontents%[current section]
% \end{frame}

\begin{frame}
  \frametitle{Security News}
  \begin{itemize}
    \item SF LR hacked \url{http://www.theverge.com/2016/11/27/13758412/hackers-san-francisco-light-rail-system-ransomware-cybersecurity-muni}
    \item Snoopers Charter \url{http://www.independent.co.uk/voices/snoopers-charter-theresa-may-online-privacy-investigatory-powers-act-a7426461.html}
    \item MD5 poisoning \url{https://blog.silentsignal.eu/2016/11/28/an-update-on-md5-poisoning/}
    \item Shodan Membership \$5 \url{https://www.shodan.io/store/member}
    \item RMCCDC is a go
    \item .mil open to hacking \url{https://krebsonsecurity.com/2016/11/dod-opens-mil-to-legal-hacking-within-limits/}
    \item windoes Priv esc via update \url{http://blog.win-fu.com/2016/11/every-windows-10-in-place-upgrade-is.html}
    \item filevault stuff \url{https://eprint.iacr.org/2012/374.pdf}
    \item more filevault stuff \url{https://www.cl.cam.ac.uk/~osc22/docs/cl_fv2_presentation_2012.pdf}
  \end{itemize}
\end{frame}

\begin{frame}
  \frametitle{A brief history of cellphones}
\begin{itemize}
  \item Analog
  \begin{itemize}
    \item 0G
    \item 1G
    \item bag phones
    \item car phones
    \item bricks
  \end{itemize}
  \item Digital
  \begin{itemize}
    \item CDMA
    \item GSM
    \item LTE
    \item 4G?
  \end{itemize}
\end{itemize}
\end{frame}

\begin{frame}
  \frametitle{Analog}
  \begin{columns}[c]
    \column{0.5\textwidth}
  \begin{itemize}
    \item 0G FM VHF half (later full) duplex (1946--2012)
    \begin{itemize}
      \item large powerful towers that covered a long range
    \end{itemize}
    \item 1G
    \begin{itemize}
      \item smaller `Cells'
      \item digital signaling --- analog voice
      \item cellphone hackers --- phone cloning and call interception
      \item 800MHz blocking on scanners in US only
    \end{itemize}
  \end{itemize}
  \column{0.5\textwidth}
  \includegraphics[width=.8\textwidth]{Motorola2950}
\end{columns}
\note[item]{0G only had 3 frequency pairs at first, everything was operator
driven, mobile telephone service Bell/Motorola}
\note[item]{this was later replaced by IMTS (1964) with the convience of \textbf{direct dial}}
\note[item]{offered by both wireline common carriers and radio common carriers}
\note[item]{IMTS had 25W at mobile station and 100--250W at base, unlike cellphones w/ 600mW}
\note[item]{IMTS:\@ limited number of customers, airtime expensive}
\note[item]{1G Advanced Mobile Phone System --- started 1983 US no longer requiried by 2/2008}
\note[item]{cloning involved recording the ESN/MDN and then adding it to another phone}
\note[item]{47cfr15.121}
\end{frame}

\begin{frame}
  \frametitle{Bag Phone Full}
  \begin{center}
    \includegraphics[width=.8\textwidth]{Motorola_Bag_Phone_Outside_Bag}
  \end{center}
\end{frame}

\begin{frame}
  \frametitle{CDMA}
  \begin{itemize}
    \item Code Division Multiple Access
    \item Propriatary tech first developed by qualcom, later standardized
    \item Used primarily in US and South Korea, later migrated to Europe and Asia
    \item 2G was cdmaOne, 3G is CDMA2000 / EVDO (data) / 1x (voice)
    \item 2G not encrypted?, 3G is
    %TODO look at encryption and evolution of CDMA, evdo etc
  \end{itemize}
  \note[item]{does not limit number of active radios}
\end{frame}


\begin{frame}
  \frametitle{GSM}
  \begin{itemize}
    \item Time Division Multiple Access (2G) and CDMA (3G)
    \item `open' standard
    \item Primarily deployed in Europe and Asia, but now deployed across world
    \item had support for encryption
    \item 2.5G is EDGE
    \item 3G is UMTS
  \end{itemize}
  \note[item]{TDMA limits number of active radios per cell}
\end{frame}

\begin{frame}
  \frametitle{LTE}
  \begin{itemize}
    \item Not 4G yet
    \item same system for both CDMA and GSM based carriers.
  \end{itemize}
\end{frame}



\begin{frame}
  \frametitle{CDMA encryption}
  \begin{itemize}
    \item CAVE (Cellular Authentication and Voice Encryption)
    \item CDMA2000 and related 3G tech uses 64 bit primary key along w/ 128 bit shared secret
  \end{itemize}
  \note[item]{primary key only used to generate shared secret which is used for signing and auth}
  \note[item]{shared secret is actually 2 64 bit keys, one for auth signatures and one for session key gen}
\end{frame}

\begin{frame}
  \frametitle{GSM encryption}
  \begin{itemize}
    \item uses A5/1 A5/2 and A5/3 stream ciphers for voice
    \item GPRS uses GEA/1, GEA/2 (vulnerable) and GEA/3 (secure?)
    \item most countries do not encrypt GPRS data for snooping porpises
    \item A5/1 has 54 bit key, originally going to be 128 but Germans
    \item A5/2 is same as A5/1 but without irregular clocking, used for export
    \item A5/3 aka KASUMI used in 3G GSM
    \item All three of these are broken.
  \end{itemize}
  \note[item]{irregular clocking: essentially randomly chooses shift registers}
\end{frame}

\begin{frame}
  \frametitle{GSM encryption}
  \begin{itemize}
    \item A5/1 is vulnerable to known-plaintext attacks with rainbow tables
    \item A5/2 is vulnerable to known-ciphertext attacks
    \item both of these can be decrypted in realtime by a 1999 era desktop PC
  \end{itemize}

  \begin{alertblock}{¡WARNING!}
    Currently, decrypting GSM traffic using these methods are illegal
  \end{alertblock}
\end{frame}

\begin{frame}
  \frametitle{Clipper Chip}
\end{frame}



\begin{frame}
  \frametitle{References}[allowframebreaks]
  \url{https://en.wikipedia.org/wiki/Mobile_Telephone_Service}
  \url{https://en.wikipedia.org/wiki/Improved_Mobile_Telephone_Service}
  \url{https://en.wikipedia.org/wiki/Advanced_Mobile_Phone_System}
  \url{https://en.wikipedia.org/wiki/0G}
  \url{https://en.wikipedia.org/wiki/1G}
  \url{http://www.arrl.org/forum/topics/view/112}
  \url{https://upload.wikimedia.org/wikipedia/en/4/46/Motorola2950.jpg}
  \url{https://upload.wikimedia.org/wikipedia/commons/0/06/Motorola_Bag_Phone_Outside_Bag.JPG}
  \url{https://en.wikipedia.org/wiki/2G}
  \url{https://en.wikipedia.org/wiki/CDMA2000}
  \url{https://www.cdg.org/resources/files/white_papers/Qualcomm_Security_Provisions_in_CDMA2000_Networks_80-W3633-1_RevA[2]_Nov2011.pdf}
  \url{https://www.scribd.com/doc/22599374/Security-Encryption-in-GSM-GPRS-CDMA}
  \url{https://en.wikipedia.org/wiki/CAVE-based_authentication}
  \url{http://www.forbes.com/sites/andygreenberg/2011/08/12/codebreaker-karsten-nohl-why-your-phone-is-insecure-by-design/\#7a6b22b72b56}
  \url{http://www.crypto.com/papers/others/a5.ps}
\end{frame}

\end{document}
